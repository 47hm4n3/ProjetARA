\documentclass[10pt]{report}

%%%%%%%%%%%%%%%%%%%%%%%%%%%%%%%%%% DEBUT DU PREAMBULE %%%%%%%%%%%%%%%%%%%%%%%%%%%%%%%%%
\title{Rapport Devoir ARA}															%%%
\usepackage[francais]{babel}														%%%
\usepackage[T1]{fontenc}															%%%
\usepackage[utf8]{inputenc}															%%%
\usepackage{fancyhdr}																%%%
\usepackage{lastpage}																%%%
\usepackage{graphicx, wrapfig, subcaption, setspace, booktabs}						%%%
\usepackage[T1]{fontenc}															%%%
\usepackage[font=small, labelfont=bf]{caption}										%%%
\usepackage[protrusion=true, expansion=true]{microtype}								%%%
\usepackage{sectsty}																%%%
\usepackage{url}																	%%%
\usepackage[colorlinks=true, urlcolor=blue, linkcolor=black]{hyperref}				%%%
\usepackage[margin=1in]{geometry}													%%%
\renewcommand\thesection{\arabic{section}}											%%%
\newcommand{\HRule}[1]{\rule{\linewidth}{#1}}										%%%
\onehalfspacing																		%%%
\setcounter{tocdepth}{5}															%%%
\setcounter{secnumdepth}{5}															%%%
\usepackage[]{array}                                                                %%%
\usepackage{mathtools}																%%%
\usepackage{underscore}
%%%%%%%%%%%%%%%%%%%%%%%%%%%%%%%%%% FIN DU PREAMBULE %%%%%%%%%%%%%%%%%%%%%%%%%%%%%%%%%%%

%--------
\usepackage[pdftex]{xcolor}
\usepackage{listings}
\usepackage{framed}
\usepackage{adjustbox}

\definecolor{javared}{rgb}{0.6,0,0} % for strings
\definecolor{javagreen}{rgb}{0.25,0.5,0.35} % comments
\definecolor{javapurple}{rgb}{0.5,0,0.35} % keywords
\definecolor{javadocblue}{rgb}{0.25,0.35,0.75} % javadoc
\definecolor{shadecolor}{rgb}{.94,.97,1}
\lstset{language=Java,
basicstyle=\ttfamily,
keywordstyle=\color{javapurple}\bfseries,
stringstyle=\color{javared},
commentstyle=\color{javagreen},
morecomment=[s][\color{javadocblue}]{/**}{*/},
numbers=left,
numberstyle=\color{black},
stepnumber=1,
numbersep=8pt,
tabsize=2,
showspaces=false,
showstringspaces=false,
columns=fullflexible,
frame = single,
breaklines=true,
postbreak=\mbox{\textcolor{red}{$\hookrightarrow$}\space}
}


%--------


%-------------------------------------------------------------------------------
% HEADER & FOOTER
%-------------------------------------------------------------------------------
\pagestyle{fancy}
\fancyhf{}
\setlength\headheight{15pt}
\fancyhead[L]{BENTAHAR \& AINAS}
\fancyhead[R]{UPMC}
%\fancyfoot[R]{Page \thepage\ / \pageref{LastPage}}
%-------------------------------------------------------------------------------
% TITLE PAGE
%-------------------------------------------------------------------------------

\begin{document}

\title{ \normalsize \textsc{\LARGE {Systèmes et Applications Répartis}\\\Large{Algorithmique Répartie Avancée}}
		\\ [2.0cm]
		\HRule{0.5pt} \\
		\LARGE \textbf{\uppercase{Devoir ARA 2017-2018\\MANET}}
		\HRule{2pt} \\ [0.5cm]
		\normalsize \vspace*{3\baselineskip}}

\author{
		Athmane BENTAHAR (3410322)\\Tacfarinas AINAS (3001048)}

\maketitle
\tableofcontents
\newpage
%-------------------------------------------------------------------------------
% Section title formatting
%-------------------------------------------------------------------------------
\sectionfont{\scshape}
%-------------------------------------------------------------------------------
% BODY
%-------------------------------------------------------------------------------
\section{Introduction}
\section{Exercice 1  Implémentation d'un MANET dans PeerSim}
\subsection{Question 1}

Initialement, un objet de la classe doit avoir \\
-une vitesse (entre min et max) qu'il peut avoir quand il est en mouvement, \\
-un temps lorsqu'il est en pause, \\
-et avoir connaissance de la longueur et largeur de la frame dans laquelle les nœuds pourront agir. \\
Il doit aussi avoir des stratégies de positions qui vont permettre au noeuds de savoir, initialement et dans l'exécution, les positions qu'ils auront. \\
Déplacement du noeud : \\
1) Lorsque le noeud n'est pas en mouvement, c'est à dire initialement où lorsqu'il est en pause, on calcule une vitesse alétoire entre min et max, puis on calcule la position destination selon la vitesse calculée et la stratégie NextDestination adoptée. \\
2) Si on a pas atteint la destination, on avance vers elle en ligne droite jusqu'à l'atteindre, puis on se met en pause et on réitère 1) \\
\subsection{Question 2}
Voici le contenu du fichier de configuration à ce point du projet :
\newline
\newline
\noindent\begin{minipage}{\textwidth}
\begin{framed}
\begin{shaded}
random.seed 5\\
network.size 3\\
simulation.endtime 100000\\
debug.config all\\
\#\#\#\#\#\#\#\#\#\#\#\#\#\#\#\#\#\#\# protocols\\
protocol.pos manet.positioning.PositionProtocolImpl\\
protocol.pos.maxspeed 100\\
protocol.pos.minspeed 50\\
protocol.pos.width 1000\\
protocol.pos.height 1000\\
protocol.pos.pause 10\\
\#\#\#\#\#\#\#\#\#\#\#\#\#\#\#\#\#\#\# initialization\\
\# initializer\\
init.initial manet.positioning.Initialize\\
init.initial.positionprotocol pos\\
\# strategies\\
initial\_position\_strategy manet.positioning.strategies.Strategy1InitNext\\
initial\_position\_strategy.positionprotocol pos\\
next\_destination\_strategy manet.positioning.strategies.Strategy1InitNext\\
next\_destination\_strategy.positionprotocol pos\\
\#\#\#\#\#\#\#\#\#\#\#\#\#\#\#\# control\\
control.graph manet.GraphicalMonitor\\
control.graph.positionprotocol pos\\
control.graph.step 2\\
control.graph.time\_slow 0.0002
\end{shaded}
\end{framed}
\end{minipage}

\subsection{Question 3}
La stratégie 1 fait mouvoir les nœuds du réseau dans l'espace compris entre [0,maxX] et [0,maxY], en générant une nouvelle position (aléatoire) aux nœuds à chaque appel.

\subsection{Question 4}
La stratégie 2 ne fait que restituer la position actuelle du nœud comme nouvelle position. Ce qui mène à une immobilité des nœuds.

\subsection{Question 5}

\noindent\begin{minipage}{\textwidth}
\begin{shaded}
\begin{lstlisting}
@Override
public void emit(Node host, Message msg) {
	for (int i = 0; i < Network.getCapacity(); i++) { // For all network
															// nodes
		node = Network.get(i);
		if (host.getID() != node.getID()) { // Except me
			if (host.getID() == msg.getIdDest()) { // I am the recipient of
														// the received message
				hostPos = (PositionProtocol) host.getProtocol(position_pid);
				nodePos = (PositionProtocol) node.getProtocol(position_pid);
				if ((hostPos.getCurrentPosition().distance(nodePos.getCurrentPosition()) <= this.getScope())) {
					EDSimulator.add(this.getLatency(), msg, node, position_pid); // Send()
				}
			}
		}
	}
}
\end{lstlisting}
\end{shaded}
\end{minipage}



\subsection{Question 6}

\noindent\begin{minipage}{\textwidth}
\begin{shaded}
\begin{lstlisting}
public class DetecterVoisinsQ6 implements NeighborProtocol, EDProtocol {

	public static final String addNeighbor_event = "ADDNEIGHBOREVENT";
	public static final String timer_event = "timer";

	private static final String PAR_PROBE = "probe";
	private static final String PAR_TIMER = "timer";
	private static final String PAR_EMITER = "emiter";

	private final int probe;
	private final int timer;
	private final int emiter;
	private final int my_pid;

	private Map<Long, Boolean> T = new HashMap<Long, Boolean>();
	private List<Long> Neighbors = new ArrayList<Long>();

	public DetecterVoisinsQ6(String prefix) {
		String tmp[] = prefix.split("\\.");
		my_pid = Configuration.lookupPid(tmp[tmp.length - 1]);
		// heartbeat=Configuration.getInt(prefix+"."+PAR_HEARTBEAT);
		probe = Configuration.getInt(prefix + "." + PAR_PROBE);
		timer = Configuration.getInt(prefix + "." + PAR_TIMER);
		emiter = Configuration.getPid(prefix + "." + PAR_EMITER);

	}

	public Object clone() {
		DetecterVoisinsQ6 v = null;
		try {
			v = (DetecterVoisinsQ6) super.clone();
			Neighbors = new ArrayList<Long>();
			T = new HashMap<Long, Boolean>();
		} catch (CloneNotSupportedException e) {
		}
		return v;
	}

	@Override
	public List<Long> getNeighbors() {
		return Neighbors;
	}

	private void addNeighbor(Long n) {
		if (!Neighbors.contains(n)) {
			Neighbors.add(n);
		}
	}

	private void removeNeighbor(Long n) {
		if (Neighbors.contains(n)) {
			Neighbors.remove(n);
		}
	}

	@Override
	public void processEvent(Node node, int pid, Object event) {
		if (pid != my_pid) {
			throw new RuntimeException("Receive Event for wrong protocol");
		}
		if (event instanceof String) {
			String ev = (String) event;
			if (ev.equals(addNeighbor_event)) {
				EmitterQ5Impl e = (EmitterQ5Impl) node.getProtocol(emiter);
				e.emit(node, new Message(node.getID(), -1, "", null, my_pid));
				EDSimulator.add(probe, addNeighbor_event, node, my_pid);

			} else {
				if (ev.equals(timer_event)) {
					for (Long l : T.keySet()) {
						if (!T.get(l)) {
							removeNeighbor(l);
						}
						T.put(l, false);
					}
					EDSimulator.add(timer, timer_event, node, my_pid);
				} else {
					EDSimulator.add(probe, addNeighbor_event, node, my_pid);
					EDSimulator.add(timer, timer_event, node, my_pid);
				}
			}
		} else {
			if (event instanceof Message) {
				Message m = (Message) event;
				addNeighbor(m.getIdSrc());
				T.put(m.getIdSrc(), true);
			}
		}
	}
}

\end{lstlisting}
\end{shaded}
\end{minipage}

\subsection{Question 7}
\subsection{Question 8}
\subsection{Question 9}
\subsection{Question 10}
Le tableau complété :\\

\begin{center}
\begin{tabular}{|l|l|l|l|l|l|}
  \hline
  Portée & SPI & SD & D($t_{end}$) & $\frac{E(t_{end})}{D(t_{end})}$ & $\frac{ED(t_{end})}{D(t_{end})}$\\
  \hline
	125 & 1 & 1 & 0 & 0 & 0\\
  \hline
	250 & 1 & 1 & 0 & 0 & 0\\
  \hline
	375 & 1 & 1 & 0 & 0 & 0\\
  \hline
	500 & 1 & 1 & 0 & 0 & 0\\
  \hline
	625 & 1 & 1 & 0 & 0 & 0\\
  \hline
	750 & 1 & 1 & 0 & 0 & 0\\
  \hline
	875 & 1 & 1 & 0 & 0 & 0\\
  \hline
	1000 & 1 & 1 & 0 & 0 & 0\\
  \hline
	125 & 3 & 3 & 0 & 0 & 0\\
  \hline
	250 & 3 & 3 & 0 & 0 & 0\\
  \hline
	375 & 3 & 3 & 0 & 0 & 0\\
  \hline
	500 & 3 & 3 & 0 & 0 & 0\\
  \hline
	625 & 3 & 3 & 0 & 0 & 0\\
  \hline
	750 & 3 & 3 & 0 & 0 & 0\\
  \hline
	875 & 3 & 3 & 0 & 0 & 0\\
  \hline
	1000 & 3 & 3 & 0 & 0 & 0\\
  \hline
\end{tabular}
\end{center}


\subsection{Question 11}
\newpage
\section{Exercice 2  Étude de protocoles de diffusion}
\subsection{Question 1}
Le tableau complété :\\

\begin{center}
\begin{tabular}{|l|l|l|}
  \hline
  Taille réseau & D($t_{end}$) & $\frac{ED(t_{end})}{D(t_{end})}$\\
  \hline
	20 & 0 & 0\\
  \hline
  	30 & 0 & 0\\
  \hline
  	40 & 0 & 0\\
  \hline
  	50 & 0 & 0\\
  \hline
  	60 & 0 & 0\\
  \hline
  	70 & 0 & 0\\
  \hline
        80 & 0 & 0\\
  \hline
  	90 & 0 & 0\\
  \hline
  	100 & 0 & 0\\
  \hline
  	120 & 0 & 0\\
  \hline
  	140 & 0 & 0\\
  \hline
  	160 & 0 & 0\\
  \hline
  	180 & 0 & 0\\
  \hline
  	200 & 0 & 0\\
  \hline
\end{tabular}
\end{center}

\subsection{Question 2}
\subsection{Question 3}
\subsection{Question 4}
\subsection{Question 5}
\subsection{Question 6}
\subsection{Question 7}
\subsection{Question 8}
\subsection{Question 9}
%-------------------------------------------------------------------------------
% REFERENCES
%-------------------------------------------------------------------------------
\newpage
\section*{Références}
\href{https://pages.lip6.fr/Jonathan.Lejeune/}{Jonathan Lejeune}, Devoir ARA, \textit{MANET.} \href{https://pages.lip6.fr/Jonathan.Lejeune/documents/enseignements/ARA/sujet\_devoir\_2017\_2018.pdf}{Énoncé}, Décembre 2017.
\newline
\newline
\end{document}
%-------------------------------------------------------------------------------
% SNIPPETS
%-------------------------------------------------------------------------------

%\begin{figure}[!ht]
%	\centering
%	\includegraphics[width=0.8\textwidth]{file_name}
%	\caption{}
%	\centering
%	\label{label:file_name}
%\end{figure}

%\begin{figure}[!ht]
%	\centering
%	\includegraphics[width=0.8\textwidth]{graph}
%	\caption{Blood pressure ranges and associated level of hypertension (American Heart Association, 2013).}
%	\centering
%	\label{label:graph}
%\end{figure}

%\begin{wrapfigure}{r}{0.30\textwidth}
%	\vspace{-40pt}
%	\begin{center}
%		\includegraphics[width=0.29\textwidth]{file_name}
%	\end{center}
%	\vspace{-20pt}
%	\caption{}
%	\label{label:file_name}
%\end{wrapfigure}

%\begin{wrapfigure}{r}{0.45\textwidth}
%	\begin{center}
%		\includegraphics[width=0.29\textwidth]{manometer}
%	\end{center}
%	\caption{Aneroid sphygmomanometer with stethoscope (Medicalexpo, 2012).}
%	\label{label:manometer}
%\end{wrapfigure}

%\begin{table}[!ht]\footnotesize
%	\centering
%	\begin{tabular}{cccccc}
%	\toprule
%	\multicolumn{2}{c} {Pearson's correlation test} & \multicolumn{4}{c} {Independent t-test} \\
%	\midrule	
%	\multicolumn{2}{c} {Gender} & \multicolumn{2}{c} {Activity level} & \multicolumn{2}{c} {Gender} \\
%	\midrule
%	Males & Females & 1st level & 6th level & Males & Females \\
%	\midrule
%	\multicolumn{2}{c} {BMI vs. SP} & \multicolumn{2}{c} {Systolic pressure} & \multicolumn{2}{c} {Systolic Pressure} \\
%	\multicolumn{2}{c} {BMI vs. DP} & \multicolumn{2}{c} {Diastolic pressure} & \multicolumn{2}{c} {Diastolic pressure} \\
%	\multicolumn{2}{c} {BMI vs. MAP} & \multicolumn{2}{c} {MAP} & \multicolumn{2}{c} {MAP} \\
%	\multicolumn{2}{c} {W:H ratio vs. SP} & \multicolumn{2}{c} {BMI} & \multicolumn{2}{c} {BMI} \\
%	\multicolumn{2}{c} {W:H ratio vs. DP} & \multicolumn{2}{c} {W:H ratio} & \multicolumn{2}{c} {W:H ratio} \\
%	\multicolumn{2}{c} {W:H ratio vs. MAP} & \multicolumn{2}{c} {\% Body fat} & \multicolumn{2}{c} {\% Body fat} \\
%	\multicolumn{2}{c} {} & \multicolumn{2}{c} {Height} & \multicolumn{2}{c} {Height} \\
%	\multicolumn{2}{c} {} & \multicolumn{2}{c} {Weight} & \multicolumn{2}{c} {Weight} \\
%	\multicolumn{2}{c} {} & \multicolumn{2}{c} {Heart rate} & \multicolumn{2}{c} {Heart rate} \\
%	\bottomrule
%	\end{tabular}
%	\caption{Parameters that were analysed and related statistical test performed for current study. BMI - body mass index; SP - systolic pressure; DP - diastolic pressure; MAP - mean arterial pressure; W:H ratio - waist to hip ratio.}
%	\label{label:tests}
%\end{table}
