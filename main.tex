\documentclass[10pt]{report}

%%%%%%%%%%%%%%%%%%%%%%%%%%%%%%%%%% DEBUT DU PREAMBULE %%%%%%%%%%%%%%%%%%%%%%%%%%%%%%%%%
\title{Rapport Devoir ARA}															%%%
\usepackage[francais]{babel}														%%%
\usepackage[T1]{fontenc}															%%%
\usepackage[utf8]{inputenc}															%%%
\usepackage{fancyhdr}																%%%
\usepackage{lastpage}																%%%
\usepackage{graphicx, wrapfig, subcaption, setspace, booktabs}						%%%
\usepackage[T1]{fontenc}															%%%
\usepackage[font=small, labelfont=bf]{caption}										%%%
\usepackage{fourier}																%%%
\usepackage[protrusion=true, expansion=true]{microtype}								%%%
\usepackage{sectsty}																%%%
\usepackage{url}																	%%%
\usepackage[colorlinks=true, urlcolor=blue, linkcolor=black]{hyperref}				%%%
\usepackage[margin=1in]{geometry}													%%%
\renewcommand\thesection{\arabic{section}}											%%%
\newcommand{\HRule}[1]{\rule{\linewidth}{#1}}										%%%
\onehalfspacing																		%%%
\setcounter{tocdepth}{5}															%%%
\setcounter{secnumdepth}{5}															%%%
\usepackage[]{array}                                                                %%%
\usepackage{mathtools}																%%%
\usepackage{underscore}																%%%
\usepackage{float}																	%%%
%%%%%%%%%%%%%%%%%%%%%%%%%%%%%%%%%% FIN DU PREAMBULE %%%%%%%%%%%%%%%%%%%%%%%%%%%%%%%%%%%
\usepackage[final]{pdfpages}
%--------
\usepackage{tcolorbox,listings}
\usepackage{framed}
\usepackage{xcolor}
\usepackage{mdframed}

\newenvironment{roundedframe}{%
  \def\FrameCommand{\tcbox[arc=5mm,colframe=red]}%
  \MakeFramed {\advance\hsize-\width \FrameRestore}}%
 {\endMakeFramed}

\definecolor{javared}{rgb}{0.6,0,0} % for strings
\definecolor{javagreen}{rgb}{0.25,0.5,0.35} % comments
\definecolor{javapurple}{rgb}{0.5,0,0.35} % keywords
\definecolor{javadocblue}{rgb}{0.25,0.35,0.75} % javadoc
\definecolor{shadecolor}{rgb}{.94,.97,1}
\lstnewenvironment{boxedlisting}{
\lstset{language=Java,
basicstyle=\linespread{0.8}\ttfamily,
keywordstyle=\color{javapurple}\bfseries,
stringstyle=\color{javared},
commentstyle=\color{javagreen},
morecomment=[s][\color{javadocblue}]{/**}{*/},
numbers=left,
numberstyle=\color{black},
stepnumber=1,
numbersep=8pt,
tabsize=2,
showspaces=false,
showstringspaces=false,
columns=fullflexible,
breakatwhitespace=true,
frame = single,
backgroundcolor = \color{blue!2},
%float,
breaklines=true,
escapeinside={(*@}{@*)},
postbreak=\mbox{\textcolor{red}{$\hookrightarrow$}\space}
}}{}

\usepackage{etoolbox}
\usepackage{pgfplots}
\pgfplotsset{compat=1.14}
\usepackage{pgfplotstable}
\pgfplotstableset{% global config, for example in the preamble
	every head row/.style={before row=\toprule,after row=\midrule},
	every last row/.style={after row=\bottomrule},
    precision = 2
}
\usepackage{eso-pic,graphicx,transparent}
%--------


%-------------------------------------------------------------------------------
% HEADER & FOOTER
%-------------------------------------------------------------------------------
\pagestyle{fancy}
\fancyhf{}
\setlength\headheight{15pt}
\fancyhead[L]{BENTAHAR \& AINAS}
\fancyhead[R]{SAR - UPMC}
\fancyfoot[C]{\thepage\ / \pageref{LastPage}}
%-------------------------------------------------------------------------------
% TITLE PAGE
%-------------------------------------------------------------------------------

\begin{document}
\title{
\normalsize \textsc{\LARGE {Systèmes et Applications Répartis}\\\Large{Algorithmique Répartie Avancée}}
		\\ [2.0cm]
		\HRule{2pt} \\
		\LARGE \textbf{\uppercase{Devoir ARA 2017-2018\\MANET}}
		\HRule{2pt} \\ [0.5cm]
		\normalsize \vspace*{5\baselineskip}}

\author{
		Athmane BENTAHAR (3410322)\\Tacfarinas AINAS (3001048)\vspace{1in}}

\AddToShipoutPictureBG*{%
  \AtPageLowerLeft{%
    \transparent{0.3}\includegraphics[width=\paperwidth,height=\paperheight]{imgs/fond}%
  }%
}
\begin{mdframed}[backgroundcolor=white!20] 
\begin{minipage}{0.48\textwidth} \begin{flushleft}
\includegraphics[scale = 0.08]{imgs/upmc.png}
\end{flushleft}\end{minipage}
\begin{minipage}{0.48\textwidth} \begin{flushright}
\includegraphics[scale = 0.08]{imgs/sar.png}
\end{flushright}\end{minipage}
	\maketitle
\end{mdframed}

\tableofcontents
\newpage
%-------------------------------------------------------------------------------
% Section title formatting
%-------------------------------------------------------------------------------
\sectionfont{\scshape}
%-------------------------------------------------------------------------------
% BODY
%-------------------------------------------------------------------------------
\section{Introduction}

\newpage

\section{Exercice 1 : Implémentation d'un MANET dans PeerSim}

\subsection{Question 1}

Initialement, un objet de la classe doit avoir :
\begin{itemize}
\item une vitesse entre [min; max] qu'il peut avoir quand il est en mouvement
\item une durée lorsqu'il est en pause
\item connaissance de la longueur et largeur de la fenêtre dans laquelle il peut se déplacer.
\end{itemize}

Il doit aussi avoir des stratégies de positionnement qui vont lui permettre de calculer, initialement et dans l'exécution, ses positions.\\

En déplacement :
\begin{enumerate}
\item Lorsque le nœud n'est pas en mouvement (état initial ou en pause) une vitesse aléatoire entre [min; max] lui est calculée, puis la position destination en fonction de vitesse et de la stratégie de déplacement adoptée.
\item Si on a pas atteint la destination, on avance vers elle en ligne droite jusqu'à l'atteindre, puis on se met en pause et on réitère 1)
\end{enumerate}

\subsection{Question 2}
Voici le contenu du fichier de configuration à ce point du devoir :

\begin{boxedlisting}
random.seed 5
network.size 10
simulation.endtime 100000
################### protocols ===========================
protocol.pos manet.positioning.PositionProtocolImpl
protocol.pos.maxspeed 100
protocol.pos.minspeed 50
protocol.pos.width 1000
protocol.pos.height 1000
protocol.pos.pause 10
################### initialization ======================
# initializer
init.initial manet.positioning.Initialize
init.initial.positionprotocol pos
# strategies
initial_position_strategy manet.positioning.strategies.Strategy1InitNext
initial_position_strategy.positionprotocol pos
next_destination_strategy manet.positioning.strategies.Strategy1InitNext
next_destination_strategy.positionprotocol pos
################ control ==============================
control.graph manet.GraphicalMonitor
control.graph.positionprotocol pos
control.graph.step 2
control.graph.time_slow 0.0002
\end{boxedlisting}

\subsection{Question 3}
La stratégie \textbf{(Strategy1InitNext)} initialise les positions des nœuds du réseau dans l'espace compris entre [0,maxX] et [0,maxY], en générant une position (aléatoire) et une vitesse initiale nulle. Aussi fait mouvoir les nœuds à chaque appel en générant une nouvelle position (aléatoire).

\subsection{Question 4}
La stratégie \textbf{(Strategy2Next)} ne fait que restituer la position actuelle du nœud comme nouvelle position. Ce qui mène à une immobilité des nœuds.

\subsection{Question 5}

\begin{boxedlisting}
@Override
public void emit(Node host, Message msg) {
	for (int i = 0; i < Network.getCapacity(); i++) { // For all network
															// nodes
		node = Network.get(i);
		if (host.getID() != node.getID()) { // Except me
			if (host.getID() == msg.getIdDest()) { // I am the recipient of the received message
				hostPos = (PositionProtocol) host.getProtocol(position_pid);
				nodePos = (PositionProtocol) node.getProtocol(position_pid);
				if ((hostPos.getCurrentPosition().distance(nodePos.getCurrentPosition()) <= this.getScope())) {
					EDSimulator.add(this.getLatency(), msg, node, position_pid); // Send()
				}
			}
		}
	}
}
\end{boxedlisting}

\subsection{Question 6}

\begin{boxedlisting}
public class NeighborProtocolImpl implements NeighborProtocol, EDProtocol {

	private static final String PAR_EMITTERPID = "emitterprotocol";
	private static final String PAR_PERIOD = "period";
	private static final String PAR_DELTA = "delta";

	private final int neighbour_pid;
	private final int emitter_pid;
	private final long period;
	private final long delta;
	private List<Long> neighbours;
	private List<Long> periodN;
	private Message msg;
	
	public NeighborProtocolImpl(String prefix) {
		String tmp[] = prefix.split("\\.");
		neighbour_pid = Configuration.lookupPid(tmp[tmp.length - 1]);
		emitter_pid = Configuration.getPid(prefix + "." + PAR_EMITTERPID);
		delta = Configuration.getInt(prefix + "." + PAR_DELTA);
		period = Configuration.getInt(prefix + "." + PAR_PERIOD);
	}

	@Override
	public List<Long> getNeighbors() {
		return this.neighbours;
	}

	public long getPeriod() {
		return this.period;
	}

	public long getDelta() {
		return this.delta;
	}

	public NeighborProtocolImpl clone() {
		try {
			neighbours = new ArrayList<Long>();
			periodN = new ArrayList<Long>();
			return (NeighborProtocolImpl) super.clone();
		} catch (Exception e) {
			System.out.println("Cloning NeighborProtocolImpl Failed !");
		}
		return null;
	}

	@Override
	public void processEvent(Node host, int pid, Object event) {
		if (neighbour_pid != pid) {
			throw new RuntimeException("Receive Event for wrong protocol");
		}
		if (event instanceof String) {
			msg = new Message(host.getID(), -1, "", MessageType.probe, neighbour_pid);
			switch ((String) event) {
			case MessageType.probe:
				((EmitterImpl) host.getProtocol(emitter_pid)).emit(host, msg);
				EDSimulator.add(period, MessageType.probe, host, neighbour_pid);
				break;
			case MessageType.timer:
				for (int i = 0; i < neighbours.size(); i++) {
					if (!periodN.contains(neighbours.get(i))) {
						neighbours.remove(i);
					}
				}
				periodN.clear();
				EDSimulator.add(delta, MessageType.timer, host, neighbour_pid);
				break;
			default:
				break;
			}
		} else if (event instanceof Message) {
			long id = ((Message) event).getIdSrc();
			if (!periodN.contains(id)) {
				periodN.add(id);
			}
			if (!neighbours.contains(id)) {
				neighbours.add(id);
			}
			
		} else {
			throw new RuntimeException("Received Event of unmatching type");
		}
	}
}
\end{boxedlisting}

\subsection{Question 7}

\begin{figure}[H]
\centering
\begin{minipage}{0.3\textwidth} \begin{flushleft}
\includegraphics[height = 0.2\textheight,width = .8\textwidth]{imgs/2.png}
 		\caption[cap1]{Capture 1}
        \label{fig:Capture 1}
\end{flushleft}\end{minipage}
\begin{minipage}{0.3\textwidth} \begin{center}
\includegraphics[height = 0.2\textheight,width = .8\textwidth]{imgs/3.png}
 		\caption[cap2]{Capture 2}
        \label{fig:Capture 2}
\end{center}\end{minipage}
\begin{minipage}{0.3\textwidth} \begin{flushright}
\includegraphics[height = 0.2\textheight,width = .8\textwidth]{imgs/4.png}
 		\caption[cap1]{Capture 3}
        \label{fig:Capture 3}
\end{flushright}\end{minipage}
\end{figure}

\subsection{Question 8}

Le tableau ci-dessous montre la connexité à termes des combinaisons des différentes stratégies dans les mêmes conditions de simulation que celles des questions précédentes (10 nœuds et une fenêtre de 1000 x 1000).({\color{red}si nous faisons baisser le nombre de nœuds à 2 par exemple, certaines des affirmations du tableau ne seront plus vraies} ou {\color{green}si nous changeons la valeur du scope à 300})\\

\begin{center}
\begin{tabular}{|l|l|l|l|l|} \hline
  SD/SPI & Strategy1InitNext & Strategy3InitNext & Strategy5Init & Strategy6Init \\ \hline
	Strategy1InitNext & Non 			& Non 			 & Non & Non \\ \hline
	Strategy2Next     & Non 			& \color{red}Oui & Oui & Oui \\ \hline
	Strategy3InitNext & \color{red}Oui  & \color{red}Oui & Non & \color{red}Oui \\ \hline
	Strategy4Next     & \color{green}Non 			& Oui 			 & Oui & Oui \\ \hline
\end{tabular}
\captionof{table}{Combinaisons SPI/SD et connexité à termes}
\end{center}

La stratégie \textbf{(Strategy3InitNext)} réduit la fenêtre à un carré centré au milieu et de coté égal à 2 x (scope - marge) où marge est le minimum entre scope et 20.\\

La stratégie \textbf{(Strategy4Next)} génère une nouvelle position du noeud en fonction des voisins aux quels il est connecté en choisissant un angle et une distance dans [0, scope] en faisant bouger un nœud à la fois.\\

La stratégie \textbf{(Strategy5Init)} réduit la fenêtre au moment de l'initialisation des positions à un rectangle entre [maxX/3,maxY/3] et [maxX,maxY] ou une position en fonction de celle de l'un des voisins sur toute la fenêtre.\\

La stratégie \textbf{(Strategy6Init)} positionne les nœuds du réseau sous forme d'une étoile centrée au milieu de la fenêtre avec deux niveaux de nœuds (d'id pairs et impairs) éloignés d'une distance de scope/2.\\

\subsection{Question 9}

\begin{boxedlisting}
public class DensityController implements Control{
	private static final String PAR_NEIGHBORPID = "neighborprotocol";

	private int neighbour_pid;
	private NeighborProtocolImpl neighbourProt;
	private List<Double> densities;
	private List<Double> standardDeviations;
	
	public DensityController(String prefix) {
		neighbour_pid = Configuration.getPid(prefix+"."+PAR_NEIGHBORPID);
		densities = new ArrayList<Double>();
		standardDeviations  = new ArrayList<Double>();
	}

	@Override
	public boolean execute() {
		densities.add(getDensity());
		standardDeviations.add(getStandardDeviation());
		return false;
	}

	public double getDensity() {
		double sum = 0;
		for(int i = 0;i < Network.size();i++) {
			neighbourProt = (NeighborProtocolImpl)Network.get(i).getProtocol(neighbour_pid);
			sum += neighbourProt.getNeighbors().size();
		}
		return sum/Network.size();
	}
	
	public double getStandardDeviation() {
		double density = getDensity();
		double sum = 0;
		for (int i = 0; i<Network.size(); i++) {
			neighbourProt = (NeighborProtocolImpl)Network.get(i).getProtocol(neighbour_pid);
			 sum += Math.pow(((double)neighbourProt.getNeighbors().size() - density),2) ;
		}
		return Math.sqrt(sum/Network.size());
	}
	
	public double getAverageDensity() {
		double sum = 0;
		for (int i = 0; i < densities.size(); i++) {
			sum += densities.get(i);
		}
		return sum/densities.size();
	}
	
	public double getAverageStandardDeviation() {
		double sum = 0;
		for (int i =0; i< standardDeviations.size(); i++) {
			sum += standardDeviations.get(i);
		}
		return sum/standardDeviations.size();
	}
	
	public double getDensityStandardDeviation() {
		double sum = 0;
		double num = getAverageDensity();
		for (int i = 0; i < densities.size(); i++) {
			sum += Math.pow((densities.get(i) - num),2);
		}
		return Math.sqrt(sum/densities.size());
	}	
}
\end{boxedlisting}

\subsection{Question 10}

Le tableau complété :\\

\begin{center}
\begin{tabular}{|l|l|l|l|l|l|} \hline
  Portée & SPI & SD & D($t_{end}$) & $\frac{E(t_{end})}{D(t_{end})}$ & $\frac{ED(t_{end})}{D(t_{end})}$\\ \hline
	125 & 1 & 1 & 1.0069 & 0.8180 & 0.1913\\ \hline
	250 & 1 & 1 & 3.7215 & 0.4628 & 0.1027\\ \hline
	375 & 1 & 1 & 8.1874 & 0.3329 & 0.0865\\ \hline
	500 & 1 & 1 & 12.6308 & 0.4206 & 0.0777\\ \hline
	625 & 1 & 1 & 18.6835 & 0.2892 & 0.0992\\ \hline
	750 & 1 & 1 & 24.5550 & 0.3486 & 0.0696\\ \hline
	875 & 1 & 1 & 31.0021 & 0.2647 & 0.0620\\ \hline
	1000 & 1 & 1 & 35.2626 & 0.2104 & 0.0545\\ \hline
	125 & 3 & 3 & 30.7339 & 0.2689 & 0.0791\\ \hline
	250 & 3 & 3 & 26.8198 & 0.2874 & 0.0691\\ \hline
	375 & 3 & 3 & 26.3156 & 0.2801 & 0.0844\\ \hline
	500 & 3 & 3 & 25.7762 & 0.3078 & 0.0803\\ \hline
	625 & 3 & 3 & 25.4211 & 0.3146 & 0.0892\\ \hline
	750 & 3 & 3 & 25.2235 & 0.2990 & 0.0855\\ \hline
	875 & 3 & 3 & 25.6281 & 0.3202 & 0.0816\\ \hline
	1000 & 3 & 3 & 25.7540 & 0.3162 & 0.0720\\ \hline
\end{tabular}
\captionof{table}{Résultats de simulations}
\end{center}

\subsection{Question 11}

L'impact de la portée (scope) sur \textbf{(Strategy1InitNext)} :\\

L'impact de la portée (scope) sur \textbf{(Strategy3InitNext)} :\\

\newpage
\section{Exercice 2 : Étude de protocoles de diffusion}
\subsection{Question 1}

Le tableau complété :
\begin{figure}[h]
\begin{minipage}[b]{0.4\textwidth} \begin{flushleft}
\centering
\begin{tabular}{|l|l|l|} \hline
  Taille réseau & D($t_{end}$) & $\frac{ED(t_{end})}{D(t_{end})}$\\ \hline
	20 & 8.7350 & 0.3568\\ \hline
  	30 & 11.2243 & 0.1926\\ \hline
  	40 & 20.1602 & 0.0860\\ \hline
  	50 & 19.6453 & 0.1433\\ \hline
  	60 & 26.0493 & 0.0885\\ \hline
  	70 & 28.2511 & 0.0403\\ \hline
    80 & 31.1596 & 0.0620\\ \hline
  	90 & 35.5816 & 0.0405\\ \hline
  	100 & 40.1069 & 0.0336\\ \hline
  	120 & 48.9275 & 0.0212\\ \hline
  	140 & 54.7881 & 0.0212\\ \hline
  	160 & 62.6688 & 0.0225\\ \hline
  	180 & 65.9515 & 0.0201\\ \hline
  	200 & 74.6832 & 0.0228\\ \hline
\end{tabular}
\captionof{table}{Résultats de simulations}
\end{flushleft}\end{minipage}
%
\begin{minipage}{0.6\textwidth} \begin{flushright}
\vspace{-7cm}
\centering
\begin{tikzpicture}
\begin{axis}[
	axis y line*=right,
    xlabel=Taille réseau,
    ylabel=D($t_{end}$),
    ]
	\addplot[color=black,mark=*] table [x=N,y=D] {
    N		D
	20		8.7350
	30		11.2243
	40		20.1602
	50		19.6453
	60		26.0493
	70		28.2511
	80		31.1596
    90		35.5816
	100		40.1069
	120		48.9275
	140		54.7881
	160		62.6688
	180		65.9515
	200		74.6832
   };\label{plot1}
\end{axis}
\begin{axis}[
	axis y line*=left,
    axis x line=none,
    ylabel=$\frac{ED(t_{end})}{D(t_{end})}$, 
    legend style={at={(0.65,0.5)},anchor=west},
    ]
    \addplot table [x=N,y=EDD] {
	N		EDD
	20		0.3568
	30		0.1926
	40		0.0860
	50		0.1433
	60		0.0885
	70		0.0403
	80		0.0620
    90		0.0405
	100		0.0336
	120		0.0212
	140		0.0212
	160		0.0225
	180		0.0201
	200		0.0228
    };\label{plot2}
    \addlegendimage{/pgfplots/refstyle=plot1}\addlegendentry{$\frac{ED(t_{end})}{D(t_{end})}$}
   	\addlegendentry{D($t_{end}$)}
\end{axis}
\end{tikzpicture}
	\caption[courbe1]{Courbe représentative}
\end{flushright}\end{minipage}
\end{figure}

Interprétation de la courbe :\\
Interprétation ...

\subsection{Question 2}

Expliquez votre démarche pour régler ce problème. Votre solution devra se faire de
manière non intrusive, ni dans le code applicatif, ni dans le code qui vous a été fourni.

\newpage
\subsection{Question 3}

\begin{boxedlisting}
public class GossipControler implements Control {
	private static final String PAR_EMITTERPID = "emitterdecoratorprotocol";
	private static final String PAR_GOSSIPPID = "gossipprotocol";
	private static final String PAR_WAVESNUMBER = "wavesnumber";
	private final int emitterdecorator_pid;
	private final int gossip_pid;
	private final int waves_number;
	private Node node;
	private int wave;
	private List<Double> atts;
	private List<Double> ERs;
	private double nbAtt;
	private boolean start = true;

	public GossipControler(String prefix) {
		emitterdecorator_pid = Configuration.getPid(prefix + "." + PAR_EMITTERPID);
		gossip_pid = Configuration.getPid(prefix + "." + PAR_GOSSIPPID);
		waves_number = Configuration.getInt(prefix + "." + PAR_WAVESNUMBER);
		wave = waves_number;
		atts = new ArrayList<Double>();
		ERs = new ArrayList<Double>();
		initialize(); // Pick a new node to broadcast first time
	}

	@Override
	public boolean execute() {
		if (wave > 0) {
			if (((EmitterDecorator) node.getProtocol(emitterdecorator_pid)).getN() == 0) { // the previous wave has finished
			if (!start) {
					double d = getAtt();
					atts.add(d); // calculate and save this wave's Att
					ERs.add(getER()); // calculate and save this wave's ER
					wave--; // decrement number of remaining waves
					resetStates(); // reset counters and start booleans
					
				}
				initialize(); // Pick a new node to broadcast
				((GossipProtocolAbstract) node.getProtocol(gossip_pid)).initiateGossip(node, wave, node.getID()); // initiate a new wave
				start = false;
			}
		} else {
			System.out.println("Moyenne Att : " + getAverageAtt());
			System.out.println("Moyenne ER : " + getAverageER());
			System.out.println("Ecart type Att : " + getAttStandardDeviation());
			System.out.println("Ecart type ER : " + getERStandardDeviation());
			return true;
		}
		return false;
	}
	
	private void resetStates() {
		for (int i = 0; i < Network.size(); i++) { // reset the first time reception boolean to true
			Node n = Network.get(i);
			GossipProtocolAbstract gpf = ((GossipProtocolAbstract) n.getProtocol(gossip_pid));
			if (gpf.getFirstRecv()) {
				gpf.setFirstRecv(false);
				gpf.setAlreadySent(false);
			}
		}
	}

	public double getAtt() {
		nbAtt = 0;
		for (int i = 0; i < Network.size(); i++) { // reset the first time reception boolean to true
			Node n = Network.get(i);
			GossipProtocolAbstract gpf = ((GossipProtocolAbstract) n.getProtocol(gossip_pid));
			if (gpf.getFirstRecv()) {
				nbAtt++;
			}
		}
		return nbAtt / ((double) Network.size());
	}

	public double getER() {
		double r = 0;
		double t = 0;
		for (int i = 0; i < Network.size(); i++) {
			Node n = Network.get(i);
			GossipProtocolAbstract gpf = ((GossipProtocolAbstract) n.getProtocol(gossip_pid));
			r += gpf.getFirstRecv() ? 1 : 0;
			t += gpf.getAlreadySent() ? 1 : 0;
		}
		return (r - t) / r;
	}
	
	public double getAverageAtt() {
		double sum = 0;
		for (int i = 0; i < atts.size(); i++) {sum += atts.get(i);}
		return sum / atts.size();
	}

	public double getAverageER() {
		double sum = 0;
		for (int i = 0; i < ERs.size(); i++) {sum += ERs.get(i);}
		return sum / ERs.size();
	}

	public double getAttStandardDeviation() {
		double sum = 0;
		double num = getAverageAtt();
		for (int i = 0; i < atts.size(); i++) {sum += Math.pow((atts.get(i) - num), 2);}
		return Math.sqrt(sum / atts.size());
	}

	public double getERStandardDeviation() {
		double sum = 0;
		double num = getAverageER();
		for (int i = 0; i < ERs.size(); i++) {sum += Math.pow((ERs.get(i) - num), 2);}
		return Math.sqrt(sum / ERs.size());
	}

	private void initialize() {
		node = Network.get(Math.abs(CommonState.r.nextInt(Network.size())));
	}
}
\end{boxedlisting}


Les résultats des test effectués avec 500 vagues et la configuration listée dans la question 1.\\

\begin{center}
\begin{tabular}{|l|l|l|l|l|}\hline
	Nb vagues & $Att_{500}$ & $ER_{500}$ & $\sigma_{Att}$ & $\sigma_{ER}$\\ \hline
	500 & 8.73 & 0.35 & 0 & 0 \\ \hline 
\end{tabular}
\end{center}

\subsection{Question 4}

\begin{figure}[H]
\begin{minipage}[b]{0.5\textwidth} \begin{flushleft}
\centering
\begin{tabular}{|l|l|l|l|l|} \hline
  Densité & $Att_{500}$ & $ER_{500}$ & $\sigma Att_{500}$ & $\sigma ER_{500}$\\ \hline
	20 & 8.73 & 0.35 & 0 & 0 \\ \hline
  	30 & 11.22 & 0.19 & 0 & 0 \\ \hline
  	40 & 20.16 & 0.08 & 0 & 0 \\ \hline
  	50 & 19.64 & 0.14 & 0 & 0 \\ \hline
  	60 & 26.04 & 0.08 & 0 & 0 \\ \hline
  	70 & 28.25 & 0.04 & 0 & 0 \\ \hline
    80 & 31.15 & 0.06 & 0 & 0 \\ \hline
  	90 & 35.58 & 0.04 & 0 & 0 \\ \hline
  	100 & 40.10 & 0.03 & 0 & 0 \\ \hline
  	120 & 48.92 & 0.02 & 0 & 0 \\ \hline
  	140 & 54.78 & 0.02 & 0 & 0 \\ \hline
  	160 & 62.66 & 0.02 & 0 & 0 \\ \hline
  	180 & 65.95 & 0.02 & 0 & 0 \\ \hline
  	200 & 74.68 & 0.02 & 0 & 0 \\ \hline
\end{tabular}
\captionof{table}{Résultats de simulations}
\end{flushleft}\end{minipage}
%
\begin{minipage}{0.5\textwidth} \begin{flushright}
\vspace{-7cm}
\centering
\begin{tikzpicture}
\begin{axis}[
	axis y line*=left,
    xlabel=Densité,
    ylabel=$Att_{500}$,
    ]
	\addplot[color=black,mark=*] table [x=D,y=Att500] {
    D		Att500
	20		8.7350
	30		11.2243
	40		20.1602
	50		19.6453
	60		26.0493
	70		28.2511
	80		31.1596
    90		35.5816
	100		40.1069
	120		48.9275
	140		54.7881
	160		62.6688
	180		65.9515
	200		74.6832
   };\label{plot1}
\end{axis}
\begin{axis}[
	axis y line*=right,
    axis x line=none,
    ylabel=$ER_{500}$, 
    legend style={at={(0.65,0.5)},anchor=west},
    ]
    \addplot table [x=D,y=ER500] {
	D		ER500
	20		0.3568
	30		0.1926
	40		0.0860
	50		0.1433
	60		0.0885
	70		0.0403
	80		0.0620
    90		0.0405
	100		0.0336
	120		0.0212
	140		0.0212
	160		0.0225
	180		0.0201
	200		0.0228
    };\label{plot2}
    \addlegendimage{/pgfplots/refstyle=plot1}\addlegendentry{$ER_{500}$}
   	\addlegendentry{$Att_{500}$}
\end{axis}
\end{tikzpicture}
	\caption[courbe1]{Courbe représentative}
\end{flushright}\end{minipage}
\end{figure}


\subsection{Question 5}
\subsection{Question 6}
\subsection{Question 7}
\subsection{Question 8}
\subsection{Question 9}
%-------------------------------------------------------------------------------
% REFERENCES
%-------------------------------------------------------------------------------
\newpage
\section*{Références}
\href{https://pages.lip6.fr/Jonathan.Lejeune/}{Jonathan Lejeune}, Devoir ARA, \textit{MANET.} \href{https://pages.lip6.fr/Jonathan.Lejeune/documents/enseignements/ARA/sujet\_devoir\_2017\_2018.pdf}{Énoncé}, Décembre 2017.\\

\newpage
%\includepdf[pages=-,pagecommand={},width=\textwidth]{enonce.pdf}
\end{document}
%-------------------------------------------------------------------------------
% SNIPPETS
%-------------------------------------------------------------------------------

%\begin{figure}[!ht]
%	\centering
%	\includegraphics[width=0.8\textwidth]{file_name}
%	\caption{}
%	\centering
%	\label{label:file_name}
%\end{figure}

%\begin{figure}[!ht]
%	\centering
%	\includegraphics[width=0.8\textwidth]{graph}
%	\caption{Blood pressure ranges and associated level of hypertension (American Heart Association, 2013).}
%	\centering
%	\label{label:graph}
%\end{figure}

%\begin{wrapfigure}{r}{0.30\textwidth}
%	\vspace{-40pt}
%	\begin{center}
%		\includegraphics[width=0.29\textwidth]{file_name}
%	\end{center}
%	\vspace{-20pt}
%	\caption{}
%	\label{label:file_name}
%\end{wrapfigure}

%\begin{wrapfigure}{r}{0.45\textwidth}
%	\begin{center}
%		\includegraphics[width=0.29\textwidth]{manometer}
%	\end{center}
%	\caption{Aneroid sphygmomanometer with stethoscope (Medicalexpo, 2012).}
%	\label{label:manometer}
%\end{wrapfigure}

%\begin{table}[!ht]\footnotesize
%	\centering
%	\begin{tabular}{cccccc}
%	\toprule
%	\multicolumn{2}{c} {Pearson's correlation test} & \multicolumn{4}{c} {Independent t-test} \\
%	\midrule	
%	\multicolumn{2}{c} {Gender} & \multicolumn{2}{c} {Activity level} & \multicolumn{2}{c} {Gender} \\
%	\midrule
%	Males & Females & 1st level & 6th level & Males & Females \\
%	\midrule
%	\multicolumn{2}{c} {BMI vs. SP} & \multicolumn{2}{c} {Systolic pressure} & \multicolumn{2}{c} {Systolic Pressure} \\
%	\multicolumn{2}{c} {BMI vs. DP} & \multicolumn{2}{c} {Diastolic pressure} & \multicolumn{2}{c} {Diastolic pressure} \\
%	\multicolumn{2}{c} {BMI vs. MAP} & \multicolumn{2}{c} {MAP} & \multicolumn{2}{c} {MAP} \\
%	\multicolumn{2}{c} {W:H ratio vs. SP} & \multicolumn{2}{c} {BMI} & \multicolumn{2}{c} {BMI} \\
%	\multicolumn{2}{c} {W:H ratio vs. DP} & \multicolumn{2}{c} {W:H ratio} & \multicolumn{2}{c} {W:H ratio} \\
%	\multicolumn{2}{c} {W:H ratio vs. MAP} & \multicolumn{2}{c} {\% Body fat} & \multicolumn{2}{c} {\% Body fat} \\
%	\multicolumn{2}{c} {} & \multicolumn{2}{c} {Height} & \multicolumn{2}{c} {Height} \\
%	\multicolumn{2}{c} {} & \multicolumn{2}{c} {Weight} & \multicolumn{2}{c} {Weight} \\
%	\multicolumn{2}{c} {} & \multicolumn{2}{c} {Heart rate} & \multicolumn{2}{c} {Heart rate} \\
%	\bottomrule
%	\end{tabular}
%	\caption{Parameters that were analysed and related statistical test performed for current study. BMI - body mass index; SP - systolic pressure; DP - diastolic pressure; MAP - mean arterial pressure; W:H ratio - waist to hip ratio.}
%	\label{label:tests}
%\end{table}
\documentclass[10pt]{report}

%%%%%%%%%%%%%%%%%%%%%%%%%%%%%%%%%% DEBUT DU PREAMBULE %%%%%%%%%%%%%%%%%%%%%%%%%%%%%%%%%
\title{Rapport Devoir ARA}															%%%
\usepackage[francais]{babel}														%%%
\usepackage[T1]{fontenc}															%%%
\usepackage[utf8]{inputenc}															%%%
\usepackage{fancyhdr}																%%%
\usepackage{lastpage}																%%%
\usepackage{graphicx, wrapfig, subcaption, setspace, booktabs}						%%%
\usepackage[T1]{fontenc}															%%%
\usepackage[font=small, labelfont=bf]{caption}										%%%
\usepackage{fourier}																%%%
\usepackage[protrusion=true, expansion=true]{microtype}								%%%
\usepackage{sectsty}																%%%
\usepackage{url}																	%%%
\usepackage[colorlinks=true, urlcolor=blue, linkcolor=black]{hyperref}				%%%
\usepackage[margin=1in]{geometry}													%%%
\renewcommand\thesection{\arabic{section}}											%%%
\newcommand{\HRule}[1]{\rule{\linewidth}{#1}}										%%%
\onehalfspacing																		%%%
\setcounter{tocdepth}{5}															%%%
\setcounter{secnumdepth}{5}															%%%
\usepackage[]{array}                                                                %%%
\usepackage{mathtools}																%%%
\usepackage{underscore}																%%%
\usepackage{float}																	%%%
%%%%%%%%%%%%%%%%%%%%%%%%%%%%%%%%%% FIN DU PREAMBULE %%%%%%%%%%%%%%%%%%%%%%%%%%%%%%%%%%%
\usepackage[final]{pdfpages}
%--------
\usepackage{tcolorbox,listings}
\usepackage{framed}
\usepackage{xcolor}
\usepackage{mdframed}

\newenvironment{roundedframe}{%
  \def\FrameCommand{\tcbox[arc=5mm,colframe=red]}%
  \MakeFramed {\advance\hsize-\width \FrameRestore}}%
 {\endMakeFramed}

\definecolor{javared}{rgb}{0.6,0,0} % for strings
\definecolor{javagreen}{rgb}{0.25,0.5,0.35} % comments
\definecolor{javapurple}{rgb}{0.5,0,0.35} % keywords
\definecolor{javadocblue}{rgb}{0.25,0.35,0.75} % javadoc
\definecolor{shadecolor}{rgb}{.94,.97,1}
\lstnewenvironment{boxedlisting}{
\lstset{language=Java,
basicstyle=\linespread{0.8}\ttfamily,
keywordstyle=\color{javapurple}\bfseries,
stringstyle=\color{javared},
commentstyle=\color{javagreen},
morecomment=[s][\color{javadocblue}]{/**}{*/},
numbers=left,
numberstyle=\color{black},
stepnumber=1,
numbersep=8pt,
tabsize=2,
showspaces=false,
showstringspaces=false,
columns=fullflexible,
breakatwhitespace=true,
frame = single,
backgroundcolor = \color{blue!2},
%float,
breaklines=true,
escapeinside={(*@}{@*)},
postbreak=\mbox{\textcolor{red}{$\hookrightarrow$}\space}
}}{}

\usepackage{etoolbox}
\usepackage{pgfplots}
\pgfplotsset{compat=1.14}
\usepackage{pgfplotstable}
\pgfplotstableset{% global config, for example in the preamble
	every head row/.style={before row=\toprule,after row=\midrule},
	every last row/.style={after row=\bottomrule},
    precision = 2
}
\usepackage{eso-pic,graphicx,transparent}
%--------


%-------------------------------------------------------------------------------
% HEADER & FOOTER
%-------------------------------------------------------------------------------
\pagestyle{fancy}
\fancyhf{}
\setlength\headheight{15pt}
\fancyhead[L]{BENTAHAR \& AINAS}
\fancyhead[R]{SAR - UPMC}
\fancyfoot[C]{\thepage\ / \pageref{LastPage}}
%-------------------------------------------------------------------------------
% TITLE PAGE
%-------------------------------------------------------------------------------

\begin{document}
\title{
\normalsize \textsc{\LARGE {Systèmes et Applications Répartis}\\\Large{Algorithmique Répartie Avancée}}
		\\ [2.0cm]
		\HRule{2pt} \\
		\LARGE \textbf{\uppercase{Devoir ARA 2017-2018\\MANET}}
		\HRule{2pt} \\ [0.5cm]
		\normalsize \vspace*{5\baselineskip}}

\author{
		Athmane BENTAHAR (3410322)\\Tacfarinas AINAS (3001048)\vspace{1in}}

\AddToShipoutPictureBG*{%
  \AtPageLowerLeft{%
    \transparent{0.3}\includegraphics[width=\paperwidth,height=\paperheight]{imgs/fond}%
  }%
}
\begin{mdframed}[backgroundcolor=white!20] 
\begin{minipage}{0.48\textwidth} \begin{flushleft}
\includegraphics[scale = 0.08]{imgs/upmc.png}
\end{flushleft}\end{minipage}
\begin{minipage}{0.48\textwidth} \begin{flushright}
\includegraphics[scale = 0.08]{imgs/sar.png}
\end{flushright}\end{minipage}
	\maketitle
\end{mdframed}

\tableofcontents
\newpage
%-------------------------------------------------------------------------------
% Section title formatting
%-------------------------------------------------------------------------------
\sectionfont{\scshape}
%-------------------------------------------------------------------------------
% BODY
%-------------------------------------------------------------------------------
\section{Introduction}

\newpage

\section{Exercice 1  Implémentation d'un MANET dans PeerSim}

\subsection{Question 1}

Initialement, un objet de la classe doit avoir :
\begin{itemize}
\item une vitesse entre [min; max] qu'il peut avoir quand il est en mouvement
\item une durée lorsqu'il est en pause
\item connaissance de la longueur et largeur de la fenêtre dans laquelle il peut se déplacer.
\end{itemize}

Il doit aussi avoir des stratégies de positionnement qui vont lui permettre de calculer, initialement et dans l'exécution, ses positions.\\

En déplacement :
\begin{enumerate}
\item Lorsque le nœud n'est pas en mouvement (état initial ou en pause) une vitesse aléatoire entre [min; max] lui est calculée, puis la position destination en fonction de vitesse et de la stratégie de déplacement adoptée.
\item Si on a pas atteint la destination, on avance vers elle en ligne droite jusqu'à l'atteindre, puis on se met en pause et on réitère 1)
\end{enumerate}

\subsection{Question 2}
Voici le contenu du fichier de configuration à ce point du devoir :

\begin{boxedlisting}
random.seed 5
network.size 10
simulation.endtime 100000
################### protocols ===========================
protocol.pos manet.positioning.PositionProtocolImpl
protocol.pos.maxspeed 100
protocol.pos.minspeed 50
protocol.pos.width 1000
protocol.pos.height 1000
protocol.pos.pause 10
################### initialization ======================
# initializer
init.initial manet.positioning.Initialize
init.initial.positionprotocol pos
# strategies
initial_position_strategy manet.positioning.strategies.Strategy1InitNext
initial_position_strategy.positionprotocol pos
next_destination_strategy manet.positioning.strategies.Strategy1InitNext
next_destination_strategy.positionprotocol pos
################ control ==============================
control.graph manet.GraphicalMonitor
control.graph.positionprotocol pos
control.graph.step 2
control.graph.time_slow 0.0002
\end{boxedlisting}

\subsection{Question 3}
La stratégie \textbf{(Strategy1InitNext)} initialise les positions des nœuds du réseau dans l'espace compris entre [0,maxX] et [0,maxY], en générant une position (aléatoire) et une vitesse initiale nulle. Aussi fait mouvoir les nœuds à chaque appel en générant une nouvelle position (aléatoire).

\subsection{Question 4}
La stratégie \textbf{(Strategy2Next)} ne fait que restituer la position actuelle du nœud comme nouvelle position. Ce qui mène à une immobilité des nœuds.

\subsection{Question 5}

\begin{boxedlisting}
@Override
public void emit(Node host, Message msg) {
	for (int i = 0; i < Network.getCapacity(); i++) { // For all network
															// nodes
		node = Network.get(i);
		if (host.getID() != node.getID()) { // Except me
			if (host.getID() == msg.getIdDest()) { // I am the recipient of the received message
				hostPos = (PositionProtocol) host.getProtocol(position_pid);
				nodePos = (PositionProtocol) node.getProtocol(position_pid);
				if ((hostPos.getCurrentPosition().distance(nodePos.getCurrentPosition()) <= this.getScope())) {
					EDSimulator.add(this.getLatency(), msg, node, position_pid); // Send()
				}
			}
		}
	}
}
\end{boxedlisting}

\subsection{Question 6}

\begin{boxedlisting}
public class NeighborProtocolImpl implements NeighborProtocol, EDProtocol {

	private static final String PAR_EMITTERPID = "emitterprotocol";
	private static final String PAR_PERIOD = "period";
	private static final String PAR_DELTA = "delta";

	private final int neighbour_pid;
	private final int emitter_pid;
	private final long period;
	private final long delta;
	private List<Long> neighbours;
	private List<Long> periodN;
	private Message msg;
	
	public NeighborProtocolImpl(String prefix) {
		String tmp[] = prefix.split("\\.");
		neighbour_pid = Configuration.lookupPid(tmp[tmp.length - 1]);
		emitter_pid = Configuration.getPid(prefix + "." + PAR_EMITTERPID);
		delta = Configuration.getInt(prefix + "." + PAR_DELTA);
		period = Configuration.getInt(prefix + "." + PAR_PERIOD);
	}

	@Override
	public List<Long> getNeighbors() {
		return this.neighbours;
	}

	public long getPeriod() {
		return this.period;
	}

	public long getDelta() {
		return this.delta;
	}

	public NeighborProtocolImpl clone() {
		try {
			neighbours = new ArrayList<Long>();
			periodN = new ArrayList<Long>();
			return (NeighborProtocolImpl) super.clone();
		} catch (Exception e) {
			System.out.println("Cloning NeighborProtocolImpl Failed !");
		}
		return null;
	}

	@Override
	public void processEvent(Node host, int pid, Object event) {
		if (neighbour_pid != pid) {
			throw new RuntimeException("Receive Event for wrong protocol");
		}
		if (event instanceof String) {
			msg = new Message(host.getID(), -1, "", MessageType.probe, neighbour_pid);
			switch ((String) event) {
			case MessageType.probe:
				((EmitterImpl) host.getProtocol(emitter_pid)).emit(host, msg);
				EDSimulator.add(period, MessageType.probe, host, neighbour_pid);
				break;
			case MessageType.timer:
				for (int i = 0; i < neighbours.size(); i++) {
					if (!periodN.contains(neighbours.get(i))) {
						neighbours.remove(i);
					}
				}
				periodN.clear();
				EDSimulator.add(delta, MessageType.timer, host, neighbour_pid);
				break;
			default:
				break;
			}
		} else if (event instanceof Message) {
			long id = ((Message) event).getIdSrc();
			if (!periodN.contains(id)) {
				periodN.add(id);
			}
			if (!neighbours.contains(id)) {
				neighbours.add(id);
			}
			
		} else {
			throw new RuntimeException("Received Event of unmatching type");
		}
	}
}
\end{boxedlisting}

\subsection{Question 7}

\begin{figure}[H]
\centering
\begin{minipage}{0.3\textwidth} \begin{flushleft}
\includegraphics[height = 0.2\textheight,width = .8\textwidth]{imgs/2.png}
 		\caption[cap1]{Capture 1}
        \label{fig:Capture 1}
\end{flushleft}\end{minipage}
\begin{minipage}{0.3\textwidth} \begin{center}
\includegraphics[height = 0.2\textheight,width = .8\textwidth]{imgs/3.png}
 		\caption[cap2]{Capture 2}
        \label{fig:Capture 2}
\end{center}\end{minipage}
\begin{minipage}{0.3\textwidth} \begin{flushright}
\includegraphics[height = 0.2\textheight,width = .8\textwidth]{imgs/4.png}
 		\caption[cap1]{Capture 3}
        \label{fig:Capture 3}
\end{flushright}\end{minipage}
\end{figure}

\subsection{Question 8}

Le tableau ci-dessous montre la connexité à termes des combinaisons des différentes stratégies dans les mêmes conditions de simulation que celles des questions précédentes (10 nœuds et une fenêtre de 1000 x 1000).({\color{red}si nous faisons baisser le nombre de nœuds à 2 par exemple, certaines des affirmations du tableau ne seront plus vraies} ou {\color{green}si nous changeons la valeur du scope à 300})\\

\begin{center}
\begin{tabular}{|l|l|l|l|l|} \hline
  SD/SPI & Strategy1InitNext & Strategy3InitNext & Strategy5Init & Strategy6Init \\ \hline
	Strategy1InitNext & Non 			& Non 			 & Non & Non \\ \hline
	Strategy2Next     & Non 			& \color{red}Oui & Oui & Oui \\ \hline
	Strategy3InitNext & \color{red}Oui  & \color{red}Oui & Non & \color{red}Oui \\ \hline
	Strategy4Next     & \color{green}Non 			& Oui 			 & Oui & Oui \\ \hline
\end{tabular}
\captionof{table}{Combinaisons SPI/SD et connexité à termes}
\end{center}

La stratégie \textbf{(Strategy3InitNext)} réduit la fenêtre à un carré centré au milieu et de coté égal à 2 x (scope - marge) où marge est le minimum entre scope et 20.\\

La stratégie \textbf{(Strategy4Next)} génère une nouvelle position du noeud en fonction des voisins aux quels il est connecté en choisissant un angle et une distance dans [0, scope] en faisant bouger un nœud à la fois.\\

La stratégie \textbf{(Strategy5Init)} réduit la fenêtre au moment de l'initialisation des positions à un rectangle entre [maxX/3,maxY/3] et [maxX,maxY] ou une position en fonction de celle de l'un des voisins sur toute la fenêtre.\\

La stratégie \textbf{(Strategy6Init)} positionne les nœuds du réseau sous forme d'une étoile centrée au milieu de la fenêtre avec deux niveaux de nœuds (d'id pairs et impairs) éloignés d'une distance de scope/2.\\

\subsection{Question 9}

\begin{boxedlisting}
public class DensityController implements Control{
	private static final String PAR_NEIGHBORPID = "neighborprotocol";

	private int neighbour_pid;
	private NeighborProtocolImpl neighbourProt;
	private List<Double> densities;
	private List<Double> standardDeviations;
	
	public DensityController(String prefix) {
		neighbour_pid = Configuration.getPid(prefix+"."+PAR_NEIGHBORPID);
		densities = new ArrayList<Double>();
		standardDeviations  = new ArrayList<Double>();
	}

	@Override
	public boolean execute() {
		densities.add(getDensity());
		standardDeviations.add(getStandardDeviation());
		return false;
	}

	public double getDensity() {
		double sum = 0;
		for(int i = 0;i < Network.size();i++) {
			neighbourProt = (NeighborProtocolImpl)Network.get(i).getProtocol(neighbour_pid);
			sum += neighbourProt.getNeighbors().size();
		}
		return sum/Network.size();
	}
	
	public double getStandardDeviation() {
		double density = getDensity();
		double sum = 0;
		for (int i = 0; i<Network.size(); i++) {
			neighbourProt = (NeighborProtocolImpl)Network.get(i).getProtocol(neighbour_pid);
			 sum += Math.pow(((double)neighbourProt.getNeighbors().size() - density),2) ;
		}
		return Math.sqrt(sum/Network.size());
	}
	
	public double getAverageDensity() {
		double sum = 0;
		for (int i = 0; i < densities.size(); i++) {
			sum += densities.get(i);
		}
		return sum/densities.size();
	}
	
	public double getAverageStandardDeviation() {
		double sum = 0;
		for (int i =0; i< standardDeviations.size(); i++) {
			sum += standardDeviations.get(i);
		}
		return sum/standardDeviations.size();
	}
	
	public double getDensityStandardDeviation() {
		double sum = 0;
		double num = getAverageDensity();
		for (int i = 0; i < densities.size(); i++) {
			sum += Math.pow((densities.get(i) - num),2);
		}
		return Math.sqrt(sum/densities.size());
	}	
}
\end{boxedlisting}

\subsection{Question 10}

Le tableau complété :\\

\begin{center}
\begin{tabular}{|l|l|l|l|l|l|} \hline
  Portée & SPI & SD & D($t_{end}$) & $\frac{E(t_{end})}{D(t_{end})}$ & $\frac{ED(t_{end})}{D(t_{end})}$\\ \hline
	125 & 1 & 1 & 1.0069 & 0.8180 & 0.1913\\ \hline
	250 & 1 & 1 & 3.7215 & 0.4628 & 0.1027\\ \hline
	375 & 1 & 1 & 8.1874 & 0.3329 & 0.0865\\ \hline
	500 & 1 & 1 & 12.6308 & 0.4206 & 0.0777\\ \hline
	625 & 1 & 1 & 18.6835 & 0.2892 & 0.0992\\ \hline
	750 & 1 & 1 & 24.5550 & 0.3486 & 0.0696\\ \hline
	875 & 1 & 1 & 31.0021 & 0.2647 & 0.0620\\ \hline
	1000 & 1 & 1 & 35.2626 & 0.2104 & 0.0545\\ \hline
	125 & 3 & 3 & 30.7339 & 0.2689 & 0.0791\\ \hline
	250 & 3 & 3 & 26.8198 & 0.2874 & 0.0691\\ \hline
	375 & 3 & 3 & 26.3156 & 0.2801 & 0.0844\\ \hline
	500 & 3 & 3 & 25.7762 & 0.3078 & 0.0803\\ \hline
	625 & 3 & 3 & 25.4211 & 0.3146 & 0.0892\\ \hline
	750 & 3 & 3 & 25.2235 & 0.2990 & 0.0855\\ \hline
	875 & 3 & 3 & 25.6281 & 0.3202 & 0.0816\\ \hline
	1000 & 3 & 3 & 25.7540 & 0.3162 & 0.0720\\ \hline
\end{tabular}
\captionof{table}{Résultats de simulations}
\end{center}

\subsection{Question 11}

L'impact de la portée (scope) sur \textbf{(Strategy1InitNext)} :\\

L'impact de la portée (scope) sur \textbf{(Strategy3InitNext)} :\\

\newpage
\section{Exercice 2  Étude de protocoles de diffusion}
\subsection{Question 1}

Le tableau complété :
\begin{figure}[h]
\begin{minipage}[b]{0.4\textwidth} \begin{flushleft}
\centering
\begin{tabular}{|l|l|l|} \hline
  Taille réseau & D($t_{end}$) & $\frac{ED(t_{end})}{D(t_{end})}$\\ \hline
	20 & 8.7350 & 0.3568\\ \hline
  	30 & 11.2243 & 0.1926\\ \hline
  	40 & 20.1602 & 0.0860\\ \hline
  	50 & 19.6453 & 0.1433\\ \hline
  	60 & 26.0493 & 0.0885\\ \hline
  	70 & 28.2511 & 0.0403\\ \hline
    80 & 31.1596 & 0.0620\\ \hline
  	90 & 35.5816 & 0.0405\\ \hline
  	100 & 40.1069 & 0.0336\\ \hline
  	120 & 48.9275 & 0.0212\\ \hline
  	140 & 54.7881 & 0.0212\\ \hline
  	160 & 62.6688 & 0.0225\\ \hline
  	180 & 65.9515 & 0.0201\\ \hline
  	200 & 74.6832 & 0.0228\\ \hline
\end{tabular}
\captionof{table}{Résultats de simulations}
\end{flushleft}\end{minipage}
%
\begin{minipage}{0.6\textwidth} \begin{flushright}
\vspace{-7cm}
\centering
\begin{tikzpicture}
\begin{axis}[
    xlabel={Taille réseau},
    ylabel={D($t_{end}$) \& $\frac{ED(t_{end})}{D(t_{end})}$},
    legend pos=north west,
    legend entries={D($t_{end}$), $\frac{ED(t_{end})}{D(t_{end})}$},
    ]
	\addplot table [x=N,y=D] {
    N		D
	20		8.7350
	30		11.2243
	40		20.1602
	50		19.6453
	60		26.0493
	70		28.2511
	80		31.1596
    90		35.5816
	100		40.1069
	120		48.9275
	140		54.7881
	160		62.6688
	180		65.9515
	200		74.6832
    };
    \addplot[color=black,mark=*] table [x=N,y=EDD] {
	N		EDD
	20		0.3568
	30		0.1926
	40		0.0860
	50		0.1433
	60		0.0885
	70		0.0403
	80		0.0620
    90		0.0405
	100		0.0336
	120		0.0212
	140		0.0212
	160		0.0225
	180		0.0201
	200		0.0228
    };
\end{axis}
\end{tikzpicture}
	\caption[courbe1]{Courbe représentative}
\end{flushright}\end{minipage}
\end{figure}

Interprétation de la courbe :\\
Interprétation ...

\subsection{Question 2}
\subsection{Question 3}
\subsection{Question 4}
\subsection{Question 5}
\subsection{Question 6}
\subsection{Question 7}
\subsection{Question 8}
\subsection{Question 9}
%-------------------------------------------------------------------------------
% REFERENCES
%-------------------------------------------------------------------------------
\newpage
\section*{Références}
\href{https://pages.lip6.fr/Jonathan.Lejeune/}{Jonathan Lejeune}, Devoir ARA, \textit{MANET.} \href{https://pages.lip6.fr/Jonathan.Lejeune/documents/enseignements/ARA/sujet\_devoir\_2017\_2018.pdf}{Énoncé}, Décembre 2017.\\

\newpage
%\includepdf[pages=-,pagecommand={},width=\textwidth]{enonce.pdf}
\end{document}
%-------------------------------------------------------------------------------
% SNIPPETS
%-------------------------------------------------------------------------------

%\begin{figure}[!ht]
%	\centering
%	\includegraphics[width=0.8\textwidth]{file_name}
%	\caption{}
%	\centering
%	\label{label:file_name}
%\end{figure}

%\begin{figure}[!ht]
%	\centering
%	\includegraphics[width=0.8\textwidth]{graph}
%	\caption{Blood pressure ranges and associated level of hypertension (American Heart Association, 2013).}
%	\centering
%	\label{label:graph}
%\end{figure}

%\begin{wrapfigure}{r}{0.30\textwidth}
%	\vspace{-40pt}
%	\begin{center}
%		\includegraphics[width=0.29\textwidth]{file_name}
%	\end{center}
%	\vspace{-20pt}
%	\caption{}
%	\label{label:file_name}
%\end{wrapfigure}

%\begin{wrapfigure}{r}{0.45\textwidth}
%	\begin{center}
%		\includegraphics[width=0.29\textwidth]{manometer}
%	\end{center}
%	\caption{Aneroid sphygmomanometer with stethoscope (Medicalexpo, 2012).}
%	\label{label:manometer}
%\end{wrapfigure}

%\begin{table}[!ht]\footnotesize
%	\centering
%	\begin{tabular}{cccccc}
%	\toprule
%	\multicolumn{2}{c} {Pearson's correlation test} & \multicolumn{4}{c} {Independent t-test} \\
%	\midrule	
%	\multicolumn{2}{c} {Gender} & \multicolumn{2}{c} {Activity level} & \multicolumn{2}{c} {Gender} \\
%	\midrule
%	Males & Females & 1st level & 6th level & Males & Females \\
%	\midrule
%	\multicolumn{2}{c} {BMI vs. SP} & \multicolumn{2}{c} {Systolic pressure} & \multicolumn{2}{c} {Systolic Pressure} \\
%	\multicolumn{2}{c} {BMI vs. DP} & \multicolumn{2}{c} {Diastolic pressure} & \multicolumn{2}{c} {Diastolic pressure} \\
%	\multicolumn{2}{c} {BMI vs. MAP} & \multicolumn{2}{c} {MAP} & \multicolumn{2}{c} {MAP} \\
%	\multicolumn{2}{c} {W:H ratio vs. SP} & \multicolumn{2}{c} {BMI} & \multicolumn{2}{c} {BMI} \\
%	\multicolumn{2}{c} {W:H ratio vs. DP} & \multicolumn{2}{c} {W:H ratio} & \multicolumn{2}{c} {W:H ratio} \\
%	\multicolumn{2}{c} {W:H ratio vs. MAP} & \multicolumn{2}{c} {\% Body fat} & \multicolumn{2}{c} {\% Body fat} \\
%	\multicolumn{2}{c} {} & \multicolumn{2}{c} {Height} & \multicolumn{2}{c} {Height} \\
%	\multicolumn{2}{c} {} & \multicolumn{2}{c} {Weight} & \multicolumn{2}{c} {Weight} \\
%	\multicolumn{2}{c} {} & \multicolumn{2}{c} {Heart rate} & \multicolumn{2}{c} {Heart rate} \\
%	\bottomrule
%	\end{tabular}
%	\caption{Parameters that were analysed and related statistical test performed for current study. BMI - body mass index; SP - systolic pressure; DP - diastolic pressure; MAP - mean arterial pressure; W:H ratio - waist to hip ratio.}
%	\label{label:tests}
%\end{table}
